\chapter{Zusammenfassung \& Reflexion}

Sortierte Sequenzen stellen eine mächtige Datenstruktur dar, insbesondere, wenn sie mit (a,b)-Bäumen als Navigationsstruktur umgesetzt werden. Die Funktionsweise von Operationen wie Finden, Einfügen und Entfernen wurde in dieser Arbeit erörtert. Die wohl wichtigste Erkenntnis stellt dar, dass diese Operationen in logarithmischer Zeit ausgeführt werden können. Die Laufzeit der Operationen entwickelt sich logarithmisch in Abhängigkeit von der Anzahl  der Elemente in der sortierten Sequenz. Diese Eigenschaft konnte auch durch die Testreihe an meiner eigenen Implementierung von (a,b)-Bäumen bestätigt werden.
\par
Besagte Implementierung führte ich in Python durch, was sich aufgrund der umkomplizierten Natur der Datentypenverwaltung in Python als richtige Entscheidung herausstellte. Gerne hätte ich C benutzt, da es schneller ist. Die Probleme und offenen Fragen bei der Umsetzung erwiesen sich jedoch als zu groß und zu umfrangreich für meinen Kenntnisstand.
\par
Während der Bearbeitung des Themas fiel mir auf, dass verhältnismäßig wenig Material über sortierte Sequenzen und (a,b)-Bäume existiert. Das Buch \textit{"`Sequential and Parallel Algorithms and Data Structures - The Basic Toolbox"'} \cite{Sanders:19} bietet den umfangreichsten Einblick in Aufbau und Funktionsweise von (a,b)-Bäumen und lieferte somit den Großteil der Informationen für diese Arbeit.
\par
Abschließend stelle ich fest, dass es sehr interessant war, sich über (a,b)-Bäume und sortierte Sequenzen zu informieren.
